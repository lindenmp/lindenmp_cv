\documentclass[11pt,a4paper,sans]{moderncv}        % possible options include font size ('10pt', '11pt' and '12pt'), paper size ('a4paper', 'letterpaper', 'a5paper', 'legalpaper', 'executivepaper' and 'landscape') and font family ('sans' and 'roman')

% moderncv themes
\moderncvstyle{banking}                             % style options are 'casual' (default), 'classic', 'oldstyle' and 'banking'
\moderncvcolor{blue}                               % color options 'blue' (default), 'orange', 'green', 'red', 'purple', 'grey' and 'black'
%\renewcommand{\familydefault}{\sfdefault}         % to set the default font; use '\sfdefault' for the default sans serif font, '\rmdefault' for the default roman one, or any tex font name
\nopagenumbers{}                                  % uncomment to suppress automatic page numbering for CVs longer than one page

% character encoding
\usepackage[utf8]{inputenc}                       % if you are not using xelatex ou lualatex, replace by the encoding you are using
\usepackage{fontawesome5}
% \usepackage{fontspec}
\usepackage{tabularx}
\usepackage{ragged2e}

% adjust the page margins
\usepackage[scale=0.9]{geometry}
% \setlength{\hintscolumnwidth}{3cm}                % if you want to change the width of the column with the dates
%\setlength{\makecvtitlenamewidth}{10cm}           % for the 'classic' style, if you want to force the width allocated to your name and avoid line breaks. be careful though, the length is normally calculated to avoid any overlap with your personal info; use this at your own typographical risks...
\usepackage{xcolor}

% personal data
\name{Linden}{Parkes, Ph.D.}
% \address{Postdoctoral Research Fellow, Department of Bioengineering}{311 Hayden Hall, University of Pennsylvania, Philadelphia, PA 19104}{}
\address{K99/R00 Postdoctoral Research Fellow, University of Pennsylvania}{}{}

\newcommand*{\customcventry}[7][.25em]{
  \begin{tabular}{@{}l}
    {\bfseries #4}
  \end{tabular}
  \hfill% move it to the right
  \begin{tabular}{l@{}}
     {\bfseries #5}
  \end{tabular} \\
  \begin{tabular}{@{}l}
    {\itshape #3}
  \end{tabular}
  \hfill% move it to the right
  \begin{tabular}{l@{}}
     {\itshape #2}
  \end{tabular}
  \ifx&#7&%
  \else{\\%
    \begin{minipage}{\maincolumnwidth}%
      \small#7%
    \end{minipage}}\fi%
  \par\addvspace{#1}}

\newcommand*{\customcventryalt}[4][.25em]{
%   \vfill\noindent
  \begin{tabular}{@{}l}
    {\bfseries #2}
  \end{tabular}
  \hfill% move it to the right
  \begin{tabular}{l@{}}
     {\itshape #3}
  \end{tabular}
  \ifx&#4&%
  \else{\\%
    \begin{minipage}{\maincolumnwidth}%
      \small#4%
    \end{minipage}}\fi%
  \par\addvspace{#1}}

\setlength{\tabcolsep}{12pt}

%----------------------------------------------------------------------------------
%            content
%----------------------------------------------------------------------------------
\begin{document}
%\begin{CJK*}{UTF8}{gbsn}                          % to typeset your resume in Chinese using CJK
%-----       resume       ---------------------------------------------------------
\makecvtitle
\vspace*{-15mm}

\begin{center}
\begin{tabular}{ c c c c }
 \faIcon{globe} lindenparkes.com & \faIcon{envelope} lindenmp@seas.upenn.edu & \faIcon{github} \href{https://github.com/lindenmp}{lindenmp} & \faIcon{twitter} \href{https://twitter.com/LindenParkes}{LindenParkes}\\
\end{tabular}
\end{center}

%%%%%%%%%%%%%%%% Academic Positions %%%%%%%%%%%%%%%%
\section{Research Interests}
I am a \textbf{computational neuroscientist} seeking to uncover the pathways that track the emergence of psychopathology. I approach this goal from a neurobiological perspective by studying how complex neural systems shape behavior and cognition, and how dysfunction in these systems predicts psychopathology.
%%%%%%%%%%%%%%%% Academic Positions %%%%%%%%%%%%%%%%
\section{Academic Positions}

{\customcventry{July 2019 - present}{University of Pennsylvania, Department of Bioengineering}{Postdoctoral Research Fellow}{Philadelphia, PA}{}
{\begin{itemize}
  \item Advisors: Prof. Dani S. Bassett, Associate Prof. Theodore D. Satterthwaite
\end{itemize}
}

{\customcventry{Sept. 2018 - Oct. 2018}{Donders Institute for Brain, Cognition and Behaviour}{Visiting Scholar}{Nijmegen, The Netherlands}{}
{\begin{itemize}
  \item Advisors: Prof. Christian Beckmann, Dr. Andre Marquand
\end{itemize}
}

%%%%%%%%%%%%%%%% Education %%%%%%%%%%%%%%%%
\section{Education}

{\customcventry{March 2014 - June 2019}{Monash University}{Doctor of Philosophy, Neuroscience \& Psychiatry}{Melbourne, Australia}{}
{\begin{itemize}
  \item Thesis: Mapping brain networks in health and mental disorder with structural and functional Magnetic Resonance Imaging
  \item Advisors: Prof. Murat Yucel, Prof. Alex Fornito, Dr. Ben Fulcher
\end{itemize}
}

{\customcventry{2009 - 2013}{Swinburne University of Technology}{Bachelor of Science (with Honors), Psychology/Psychophysiology}{Melbourne, Australia}{}
{\begin{itemize}
  \item Thesis: Mapping language processes using Magnetoencephalography.
  \item Advisor: Associate Prof. Conrad Perry
  \item Honors: First Class. Dux
\end{itemize}
}

%%%%%%%%%%%%%%%% Funding %%%%%%%%%%%%%%%%
% \section{Funding (pending)}
% \section{Funding (awarded)}
\section{Funding}
\subsection{Career Transition Awards}
{\customcventryalt{K99/R00 Pathway to Independence Award}{Sept. 2021 - 2026}{National Institute of Mental Health (NIMH)}
{\begin{itemize}
  \item Project: Developing prognostic neuroimaging biomarkers of the psychosis spectrum using network control theory
  \item \$1,424,194 USD
\end{itemize}
}

\subsection{Fellowships \& Scholarships}

{\customcventryalt{Young Investigator Award}{Jan. 2021 - Jan. 2023}{Brain \& Behavior Research Foundation}{}{}
{\begin{itemize}
  \item Project: Hybrid neurodevelopmental normative models for psychosis
  \item \$70,000 USD
\end{itemize}
}

{\customcventryalt{Monash University Postgraduade Publication Award}{2018}{Monash University}{}{}
{\begin{itemize}
  \item \$6,300 AUD
\end{itemize}
}

{\customcventryalt{Monash University Graduate Research Scholarship}{2014 - 2018}{Monash University}{}{}
{\begin{itemize}
  \item \$20,000 AUD
\end{itemize}
}

{\customcventryalt{Australian Postgraduate Award Research Scholarship}{2014 - 2018}{Australian Government}{}{}
{\begin{itemize}
  \item \$91,000 AUD
\end{itemize}
}

\subsection{Grants}
{\customcventryalt{Innovations Connections Grant}{2016 - 2017}{Department of Industry, Innovation and Science, Australia}{}{}
{\begin{itemize}
  \item \$50,000 AUD
  \item Associate investigator
\end{itemize}
}

% {\customcventryalt{Swartz Foundation}{2020 - 2021}{}{}{}
% {\begin{itemize}
%   \item \$75,000 USD
%   \item Associate investigator
% \end{itemize}
% }

\subsection{Travel Awards}
{\customcventryalt{Abstract Merit Award}{2021}{Organization for Human Brain Mapping}{}{}
{\begin{itemize}
  \item virtual, no monetary component
\end{itemize}
}

{\customcventryalt{Abstract Merit Award}{2020}{Organization for Human Brain Mapping}{}{}
{\begin{itemize}
  \item \$3,000 USD
\end{itemize}
}

{\customcventryalt{Donders-Monash Erasmus Travel Award}{2018}{Donders Institute for Brain, Cognition and Behaviour | Monash University}{}{}
{\begin{itemize}
  \item \$3,200 AUD
\end{itemize}
}

{\customcventryalt{Future Leaders Travel Award}{2015}{Monash Institute of Cognitive and Clinical Neurosciences}{}{}
{\begin{itemize}
  \item \$5,000 AUD
\end{itemize}
}

%%%%%%%%%%%%%%%% Select Publications %%%%%%%%%%%%%%%%
\section{Select Publications (9 of 31)}
\cvitem{}{\emph{Citations = 835, h-index = 15, i10-index = 18}}}
\cvitem{}{\emph{For a complete list of my publications and preprints see my \href{https://scholar.google.com.au/citations?user=JT4AhnoAAAAJ&hl=en}{\textcolor{blue}{Google Scholar}}}}

\subsection{Postdoc papers}
\cvitem{}{1. \textbf{Parkes L}, Moore TM, Calkins ME, Cieslak M, Roalf DR, Wolf DH, Gur RC, Gur RE, Satterthwaite TD \& Bassett DS (2021). Network controllability in transmodal cortex predicts positive psychosis spectrum symptoms. \href{https://www.sciencedirect.com/science/article/pii/S0006322321011756}{\emph{\textcolor{blue}{Biological Psychiatry}}}.}

\cvitem{}{2. \textbf{Parkes L}, Moore TM, Calkins ME, Cook PA, Cieslak M, Roalf DR, Wolf DH, Gur RC, Gur RE, Satterthwaite TD \& Bassett DS (2021). Transdiagnostic dimensions of psychopathology explain individuals' unique deviations from normative neurodevelopment in brain structure. \href{https://www.nature.com/articles/s41398-021-01342-6}{\emph{\textcolor{blue}{Translational Psychiatry}}}.}

\cvitem{}{3. \textbf{Parkes L}, Satterthwaite TD \& Bassett DS (2020). Towards precise resting-state fMRI biomarkers in psychiatry: synthesizing developments in transdiagnostic research, dimensional models of psychopathology, and normative neurodevelopment. \href{https://www.sciencedirect.com/science/article/pii/S0959438820301616}{\emph{\textcolor{blue}{Current Opinion in Neurobiology}}}.}
{\begin{itemize}
  \item Invited opinion piece
\end{itemize}
}

\subsection{PhD papers}
\cvitem{}{4. \textbf{Parkes L}, Tiego J, Aquino K, Braganza L, Chamberlain SR, Fontenelle L, Harrison BJ, Lorenzetti V, Paton B, Razi A, Fornito A, \& Yucel M (2019). Transdiagnostic variations in impulsivity and compulsivity in obsessive-compulsive disorder and gambling disorder correlate with effective connectivity in cortical-striatal-thalamic-cortical circuits. \href{https://www.sciencedirect.com/science/article/pii/S1053811919306585}{\emph{\textcolor{blue}{NeuroImage}}}.}

\cvitem{}{5. \textbf{Parkes L}, Fulcher B, Yucel M, \& Fornito A (2018). An evaluation of the efficacy, reliability, and sensitivity of motion correction strategies for resting-state functional MRI. \href{https://www.sciencedirect.com/science/article/pii/S1053811917310972}{\emph{\textcolor{blue}{NeuroImage}}}.}
{\begin{itemize}
  \item \href{https://www.journals.elsevier.com/neuroimage/most-cited-articles}{\emph{\textcolor{blue}{The 4th most cited paper in NeuroImage since 2018}}}
\end{itemize}
}

\cvitem{}{6. \textbf{Parkes L}, Fulcher B, Yucel M, \& Fornito A (2017). Transcriptional signatures of connectomic subregions of the human striatum. \href{https://onlinelibrary.wiley.com/doi/full/10.1111/gbb.12386}{\emph{\textcolor{blue}{Genes, Brain and Behavior}}}.}
{\begin{itemize}
  \item Amongst the top 20 downloaded from the journal in 2017
\end{itemize}
}

\cvitem{}{7. *Prochazkova L, \textbf{*Parkes L}, Dawson A, Youssef G, Ferreira GM, Lorenzetti V, Segrave RA, Fontenelle LF, \& Yucel M (2017). Unpacking the role of self-reported compulsivity and impulsivity in obsessive-compulsive disorder. \href{https://www.cambridge.org/core/journals/cns-spectrums/article/abs/unpacking-the-role-of-selfreported-compulsivity-and-impulsivity-in-obsessivecompulsive-disorder/3AE081173C6A8931ACDB7E779630DAD0}{\emph{\textcolor{blue}{CNS spectrums}}}.}
\textit{*These authors contributed equally}}

\subsection{Undergraduate papers}
\cvitem{}{8. \textbf{Parkes L}, Perry C, \& Goodin P (2016). Examining the N400m in affectively negative sentences. A magnetoencephalography study. \href{https://onlinelibrary.wiley.com/doi/abs/10.1111/psyp.12601}{\emph{\textcolor{blue}{Psychophysiology}}}.}

\subsection{Book chapters}
\cvitem{}{9. Segrave RA, Hendrikse J, \& \textbf{Parkes L}, (2019). DBS, TMS and tDCS for obsessive compulsive disorder. In \textit{A Transdiagnostic Approach to Obsessions, Compulsions and Related Phenomena}. Cambridge University Press}

%%%%%%%%%%%%%%%% Teaching Experience %%%%%%%%%%%%%%%%
\section{Teaching Experience}
{\customcventry{2020}{University of Pennsylvania, Department of Bioengineering}{Teacher's Assistant}{Philadelphia, PA}{}
{\begin{itemize}
  \item Class: Network Neuroscience
  \item Course evaluation score: 3.57/4
\end{itemize}
}

{\customcventry{2019}{University of Pennsylvania, Department of Bioengineering}{Guest Lecturer}{Philadelphia, PA}{}
{\begin{itemize}
  \item Class: Network Neuroscience
\end{itemize}
}

{\customcventry{2017 - 2018}{Monash University}{Guest Lecturer}{Melbourne, Australia}{}
{\begin{itemize}
  \item Class: Neuroscience Methods
\end{itemize}
}

{\customcventry{2014 - 2015}{Swinburne University}{Recitation Tutor}{Melbourne, Australia}{}
{\begin{itemize}
  \item Class: Undergraduate Psychology
\end{itemize}
}

{\customcventry{2013}{Swinburne University}{Recitation Tutor}{Melbourne, Australia}{}
{\begin{itemize}
  \item Class: Undergraduate Physiology
\end{itemize}
}

%%%%%%%%%%%%%%%% Research Employment %%%%%%%%%%%%%%%%
\section{Research Employment}

{\customcventry{2018}{Monash Biomedical Imaging}{Research Assisstant}{Melbourne, Australia}{}
{\begin{itemize}
  \item Analysis of positron emission tomography (PET) data
\end{itemize}
}

{\customcventry{2016 - 2017}{Torus Games \& Cogstate}{Research Engineer}{Melbourne, Australia}{}
{\begin{itemize}
  \item Developed gamified cognitive tests for neuroscience research
\end{itemize}
}

{\customcventry{2013}{Swinburne Univerisity}{Magnetoencephalography Technician}{Melbourne, Australia}{}
{\begin{itemize}
  \item Collection, preprocessing, and analysis of Magnetoencephalography (MEG) data
\end{itemize}
}

%%%%%%%%%%%%%%%% Open Science Contributions %%%%%%%%%%%%%%%%
\section{Open Science Contributions}
\subsection{Toolkits}
{\customcventryalt{Network Control}{\faIcon{python} Python}{\url{https://github.com/BassettLab/control_package}}}
{\customcventryalt{Predictive Clinical Neuroscience}{\faIcon{python} Python}{\url{https://github.com/amarquand/PCNtoolkit}}}

\subsection{Reproducibility}
{\customcventryalt{\url{https://github.com/lindenmp/neurodev_cs_predictive}}{\faIcon{python} Python}{}}
{\begin{itemize}
  \item Code to reproduce results presented in \href{https://www.sciencedirect.com/science/article/pii/S0006322321011756}{\emph{\textcolor{blue}{Parkes et al. (2021) Biological Psychiatry}}}
\end{itemize}
}

{\customcventryalt{\url{https://github.com/lindenmp/normative_neurodev_cs_t1}}{\faIcon{python} Python}{}}
{\begin{itemize}
  \item Code to reproduce results presented in \href{https://www.nature.com/articles/s41398-021-01342-6}{\emph{\textcolor{blue}{Parkes et al. (2021) Translational Psychiatry}}}
\end{itemize}
}

{\customcventryalt{\url{https://github.com/lindenmp/rs-fMRI}}{\faIcon{code} Matlab}{}}
{\begin{itemize}
  \item Code to reproduce results presented in \href{https://www.sciencedirect.com/science/article/pii/S1053811917310972}{\emph{\textcolor{blue}{Parkes et al. (2018) NeuroImage}} and \href{https://www.sciencedirect.com/science/article/pii/S1053811919306585}{\emph{\textcolor{blue}{Parkes et al. (2019) NeuroImage}}} }
\end{itemize}
}

%%%%%%%%%%%%%%%% Awards %%%%%%%%%%%%%%%%
% \section{Awards}

%%%%%%%%%%%%%%%% Presentations %%%%%%%%%%%%%%%%
\section{Presentations}
\subsection{Oral}

{\customcventryalt{Review and comparison of retrospective motion correction strategies}{2021}{\href{https://crossinvalidation.com/2021/07/19/announcing-quality-conversations-webinar-series-on-niqc-open-science-etc/}{\emph{\textcolor{blue}{Quality Conversations webinar series}}} by Dr. Pradeep Raamana}
{\begin{itemize}
  \item Invited talk
  \item Available on \href{https://youtu.be/bo2AFvJ5mYU}{\textcolor{blue}{YouTube}}
\end{itemize}
}

{\customcventryalt{Network controllability in transmodal cortex predicts positive psychosis spectrum symptoms}{2021}{Organization for Human Brain Mapping}
{\begin{itemize}
  \item 3-minute summary available on \href{https://youtu.be/1sE8RMU1Kkg}{\textcolor{blue}{YouTube}}
\end{itemize}
}

{\customcventryalt{Network Neuroscience}{2021}{DataPhilly}
{\begin{itemize}
  \item Invited talk
  \item Available on \href{https://youtu.be/VdR5KmZ3OIU}{\textcolor{blue}{YouTube}}
\end{itemize}
}

{\customcventryalt{Average controllability better predicts cognition when compared to strength}{2020}{Organization for Human Brain Mapping}
{\begin{itemize}
  \item Symposium
\end{itemize}
}

{\customcventryalt{Psychopathology explain individual’s unique deviations from normative neurodevelopment}{2020}{Organization for Human Brain Mapping}
}

{\customcventryalt{Dimensional psychiatry in corticostriatal circuits: lessons learnt from resting-state fMRI data}{2018}{University of Pennsylvania}
{\begin{itemize}
  \item Invited talk
\end{itemize}
}

{\customcventryalt{Transdiagnostic biomarkers in psychiatry}{2018}{Centre of Excellence for Integrative Brain Function, Melbourne, Australia}
{\begin{itemize}
  \item Invited talk
\end{itemize}
}

{\customcventryalt{Confounds in rs-fMRI processing}{2016}{Swinburne Univerisity, Melbourne, Australia}
{\begin{itemize}
  \item Invited talk
\end{itemize}
}

{\customcventryalt{Transcriptional signatures of connectomic subregions of the human striatum}{2015}{Students of Brain Research, Melbourne, Australia}
}

{\customcventryalt{Examining the N400m in affectively negative sentences. A magnetoencephalography study}{2013}{Australasian Cognitive Neuroscience Conference}
}

\subsection{Posters (first-author)}
{\customcventryalt{Network controllability in transmodal cortex predicts positive psychosis spectrum symptoms}{2021}{Organization for Human Brain Mapping}}
{\customcventryalt{Network controllability in transmodal cortex predicts positive psychosis spectrum symptoms}{2021}{Society of Biological Psychiatry}}
{\customcventryalt{Psychopathology explain individual’s unique deviations from normative neurodevelopment}{2020}{Organization for Human Brain Mapping}
{\customcventryalt{Impulsivity and compulsivity correlate with effective connectivity in corticostriatal circuits}{2019}{Organization for Human Brain Mapping}}
{\customcventryalt{Impulsivity and compulsivity correlate with effective connectivity in corticostriatal circuits}{2018}{Australasian Cognitive Neuroscience Conference}}
{\customcventryalt{Evaluating the efficacy and sensitivity of motion correction strategies for rs-fMRI}{2018}{Organization for Human Brain Mapping}}
{\customcventryalt{Efficacy, reliability, and sensitivity of motion correction strategies for resting-state functional MRI}{2017}{IEEE International Symposium on Biomedical Imaging}}
{\customcventryalt{Efficacy, reliability, and sensitivity of motion correction strategies for resting-state functional MRI}{2017}{Students of Brain Research}}

%%%%%%%%%%%%%%%% Academic Service %%%%%%%%%%%%%%%%
\section{Academic Service}
\subsection{Journal Peer Review}
{\begin{itemize}
  \item Nature Protocols, NeuroImage, Human Brain Mapping, Network Neuroscience, Scientific Reports, Biological Psychiatry, Psychological Medicine, Neuropsychologia, Developmental Cognitive Neuroscience, NeuroImage: Clinical, Psychiatry Research: Neuroimaging, Harvard Review of Psychiatry, International Gambling Studies, Journal of Cerebral Blood Flow \& Metabolism
\end{itemize}
}

\subsection{Committees}

{\customcventryalt{Organization for Human Brain Mapping, Open Science Special Interest Group}{2022 - 2023}{Treasurer}}
{\customcventryalt{Organization for Human Brain Mapping, Student and Postdoc Special Interest Group}{2019 - 2021}{Treasurer}}
{\customcventryalt{Australasian Cognitive Neuroscience Society, Early Career Researchers Committee}{2017}{Committee Member}}
{\customcventryalt{Australasian Cognitive Neuroscience Society, Executive Committee}{2017}{ECR Representative}}
{\customcventryalt{Students of Brain Research}{2016}{Treasurer}}

\subsection{Supervision \& Mentorship}
{\customcventryalt{Ashlea Segal}{2018 - present}{Graduate Student, Monash University}}
{\customcventryalt{Tayla Currie}{2018}{Undergraduate Honors Student, Monash University}}
{\customcventryalt{John Fallon}{2017}{Undergraduate Honors Student, Monash University}}
{\customcventryalt{Luisa Prochazkova}{2016}{International Visiting Scholar, Monash University}}
{\customcventryalt{Kristina Sabaroedin}{2016}{Undergraduate Honors Student, Monash University}}
{\customcventryalt{Lauren Den Ouden}{2016}{Undergraduate Honors Student, Monash University}}
{\customcventryalt{Stuart Oldham}{2016}{Undergraduate Honors Student, Monash University}}
{\customcventryalt{Danielle Amiet}{2016}{Undergraduate Honors Student, Monash University}}

\subsection{Outreach \& Community Engagement}
{\customcventryalt{Neuroimaging Best Practices Beyond Open Science}{2021}{Organization for Human Brain Mapping, Student and Postdoc Special Interest Group}}
{\begin{itemize}
  \item Moderator
\end{itemize}
}

{\customcventryalt{Link with Mentors}{2021}{Organization for Human Brain Mapping, Student and Postdoc Special Interest Group}}
{\begin{itemize}
  \item Moderator
\end{itemize}
}

{\customcventryalt{International Mentoring Programme}{2021}{Organization for Human Brain Mapping, Student and Postdoc Special Interest Group}}
{\begin{itemize}
  \item Mentor
\end{itemize}
}

{\customcventryalt{Network Control Theory for Neuroscientists, Education Workshop}{2020}{Organization for Human Brain Mapping}}
{\begin{itemize}
  \item Organizer, presenter
\end{itemize}
}

{\customcventryalt{Link with Mentors}{2020}{Organization for Human Brain Mapping, Student and Postdoc Special Interest Group}}
{\begin{itemize}
  \item Moderator
\end{itemize}
}

{\customcventryalt{NeuroDay}{2018}{Methodist Ladies' College, Melbourne, Australia}}
{\begin{itemize}
  \item Organizer, presenter
\end{itemize}
}

{\customcventryalt{BMH Mentor Forum}{2018}{Brain \& Mental Health Research Hub, Monash University}}
{\begin{itemize}
  \item Organizer, mentor
\end{itemize}
}
{\customcventryalt{MBI Student Forum}{2014 - 2015}{Monash Biomedical Imaging, Monash University}}
{\begin{itemize}
  \item Organizer, mentor, presenter
\end{itemize}
}

\end{document}
